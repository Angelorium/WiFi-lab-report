\section{Appendix}
\label{sec:appendix}

The following are the python scripts used to automate the tests

\begin{lstlisting}[language=Python]
import subprocess, time
import numpy as np

n_times = 10
ip_address = "192.168.1.98" # May vary
bitrates=[]
print("\n")

for i in range(n_times):
	bitrate=subprocess.check_output(f"(iperf3 -c {ip_address} | grep receiver | tr -s ' ' | cut -d ' ' -f 7)",shell=True).decode().strip("\n") 
        print(f"#{i+1} bitrate: {bitrate} Mbps")
	bitrates.append(float(bitrate))
	time.sleep(1)

print("\n")

min = round(np.min(bitrates),2)
print(f"min: {min} Mbps")
max = round(np.max(bitrates),2)
print(f"max: {max} Mbps")
avg = round(np.mean(bitrates),2)
print(f"avg: {avg} Mbps")
std = round(np.std(bitrates),2)
print(f"std: {std} Mbps")

print("\n")
\end{lstlisting}
\captionof{lstlisting}{Script for TCP tests}

\vspace{0.5 cm}

For UDP, the script is the same, with the difference in the iperf3 command, in which has been used:
\begin{lstlisting}[language=bash]
iperf3 -c {indirizzo_IP} -u -l 1472 -b 1G
\end{lstlisting}

\vspace{0.5 cm}
